\documentclass[12pt,a4paper]{ctexbook} % 使用ctexbook文档类

% openany
\usepackage{graphicx} % For including images
\usepackage{tikz} %For drawing pictures
\usepackage{array}
\usepackage{tabularx} % Both are for tabular
\usepackage{subcaption}
\usepackage{hyperref} % For hyperlinks in the document
\usepackage{fancyhdr} % For custom headers and footers
\usepackage{amsthm} % For theorem environments
\usepackage{amsmath} % For additional math symbols
\usepackage{amssymb} % For additional math symbols
\usepackage{amsfonts} % For \mathbb
\usepackage{fontspec} % For font selection
\usepackage{titlesec} % For customizing section titles
\usepackage{enumitem}
\usepackage[dvipsnames]{xcolor} % 使用 xcolor 包并启用更多预定义颜色
\usepackage{wrapfig} % 用于图片环绕
\usepackage{float} % 导入 float 包

\setmainfont{Times New Roman}
%\newfontfamily\kaishu{Kaiti SC} % 楷体
%\newfontfamily\heiti{Heiti SC} % 黑体
%\newfontfamily\fangsong{STFangsong} % 仿宋
%\newfontfamily\songti{Songti SC} % 宋体

% 设置中文标点符号后自动添加空格
%\CJKsetecglue{,.  !?:;}

\hypersetup{
    colorlinks=true,
    linkcolor=blue,
    filecolor=magenta,
    urlcolor=cyan,
}

%\theoremstyle{upright}
\newcounter{theorem}[section]
\renewcommand{\thetheorem}{\thesection.\arabic{theorem}}

% Define a new proof environment with Fangsong font
\newenvironment{pf}[1][证明]{\par\noindent{\kaishu\textbf{#1.} }\setmainfont{STFangsong}\normalfont}{\hfill\ensuremath{\Box}\par}




\renewcommand{\thetheorem}{\thesection.\arabic{theorem}}

\newtheorem{theoreminner}[theorem]{定理}
\newtheorem*{theorem*}{定理}
\newtheorem{definition}[theorem]{定义}
\newtheorem{lemma}[theorem]{引理}
\newtheorem{proposition}[theorem]{命题}
\newtheorem{corollary}[theorem]{推论}
\newtheorem{example}[theorem]{例}
\newtheorem{remark}[theorem]{注}
\newtheorem{exercise}[theorem]{练习}
\newtheorem{conjecture}[theorem]{猜想}
\newtheorem*{remark*}{注}
\newtheorem*{conjecture*}{猜想}
\newtheorem*{example*}{例}
\newcommand{\df}[1]{\textcolor{Maroon}{#1}}

% 设置图片按照节标号
\numberwithin{figure}{section}

% Define the new problem environment
\newtheoremstyle{problemstyle} % 定义新的样式
  {3pt} % 上方空白
  {3pt} % 下方空白
  {\kaishu} % 正文字体
  {} % 缩进量
  {\bfseries} % 头部字体
  {} % 头部后标点
  { } % 头部后空格
  {\thmname{#1}\thmnumber{ #2}.\thmnote{ #3}\newline} % 头部格式,标题后换行

\theoremstyle{problemstyle}
\newtheorem{problem}{}[section]

\renewenvironment{proof}[1][证明]{\par\noindent{\kaishu\textbf{#1.} }\normalfont\fontspec{STFangsong}}{\hfill\ensuremath{\Box}\par}
% Customize section titles to add the "§" symbol and independent numbering
\titleformat{\section}[block]
  {\normalfont\Large\bfseries}
  {\S\arabic{section}}
  {1em}
  {}

% Customize subsection titles with independent numbering
\titleformat{\subsection}[block]
  {\normalfont\large\bfseries}
  {\thesection.\arabic{subsection}}
  {1em}
  {}



% Redefine section and subsection numbering to be independent
\renewcommand{\thesection}{\arabic{section}}
\renewcommand{\thesubsection}{\thesection.\arabic{subsection}}
% Define a new proof environment with STFangsong font
\renewenvironment{proof}[1][证明]{\par\noindent{\kaishu\textbf{#1.} }\normalfont\CJKfamily{zhfs}}{\hfill\ensuremath{\Box}\par}

%% Define a new remark* environment with SimHei for "注:" and Kaiti SC for the content, without using the theorem counter
%\newenvironment{remark*}[1][注记.]{\par\noindent{\textbf{#1} }\kaishu}{\par}


\hypersetup{
    colorlinks=true,
    linkcolor=blue,
    filecolor=magenta,
    urlcolor=cyan,
}

\numberwithin{equation}{section} % 按 section 编号公式


\title{{\kaishu\fontsize{48}{60}\selectfont 组~合~交~换~代~数~讲~义}} % Set the title in Kaiti font and large size
\author{\bf 丁相钰}
\date{\today}

\begin{document}

% Title page
\maketitle

\section{Rings and ideals \textrightarrow{} Krull dimension}
	\begin{definition}
		$(R,+,\cdot)$ 满足如下三个条件被称为一个环: $ (R,+) $ 是一个交换群; $ (R,\cdot) $ 是一个含幺半群; $ R $ 满足分配律。
	\end{definition}
 	\begin{definition}
		非零元素$ a \in R$被称为零化子如果存在非零元素$ b $使得$ ab=0。 $
	\end{definition}
    \begin{definition}
    	非零元素$ u \in R$被称为单位元如果存在非零元素$ v $使得$ uv=1。 $
    \end{definition}
    \begin{definition}
    环$ R $的子集$ F $是有吸收性的, 如果对于 $ \forall U\subset R$, 有$ U\cdot F\subset F $。
    \end{definition}
    \begin{definition}
    	环$ R $的子集$ S $是有乘性的, 如果$ 1\in S $, 对于$ \forall x,y\in S$, 那么$ xy\in S$。
    \end{definition} 
    \begin{definition}
    	环$ R $的子集$ I $被称作理想,如果$ I $ 是一个加性子群并且具有吸收性($ IR\subset I $)。
    \end{definition}
    
    \begin{definition}\label{prime-ideal}
    	$ \mathfrak{p} $是环$ R $的一个素理想,如果$ xy\in \mathfrak{p} $, 那么有$ x\in \mathfrak{p} $或者$ y\in \mathfrak{p} $。 等价于$ R- \mathfrak{p} $ 是可乘的。
    \end{definition}
    \begin{definition}\label{domin}
    	$ R $是一个整环, 那么如下条件等价:$ R $没有零因子; <0>是素理想;$ R \setminus <0> $ 可乘。
    \end{definition}
    \begin{definition}
    	$ R $是一个域, 那么如下条件等价:$ R $的非零元素都是单位元; $<0>$ 是唯一素理想; $ R $无真理想。
    \end{definition}
    \begin{proposition}\label{prop-prime-multi}
    	对于$ \phi:R\rightarrow R^{\prime} $, $ T\subset R^{\prime} $, $ S:=\phi^{-1}(T) $, 如果$ T $是可乘的或者是素理想, 那么$ \phi^{-1}(T) $分别也是可乘的或者是素理想, 反之如果这个映射是满的也成立。
    \end{proposition}
    \begin{proof}
    	令$ S=\phi^{-1}(T) $, 因$ T $可乘, $ \phi(1)=1 $, 那么$ 1\in S $。 对于$ x,y\in S $, 因$\phi(xy) =\phi(x)\phi(y)\in T $, 那么$ xy\in S $, 所以$ S $可乘。 反之如果$ \phi $是满射, 那么$ x^{\prime}=\phi(x) $, 对任意的$ x\in S, x^{\prime}\in T $。 从而$ \phi(1)=1 $, $ \phi(x)\phi(y)=\phi(xy)\in S $, 得证。
    	
    	设$ \mathfrak{p} \subset R$是素理想, $ \phi(\mathfrak{p})=\mathfrak{p} ^{\prime}\subset R^{\prime}$。 由定义\ref{prime-ideal}和上述讨论可知, $ R-\mathfrak{p} $可乘当且仅当 $ R^{\prime}-\mathfrak{p}^{\prime} $可乘, 第二部分得证。
    \end{proof}
     \begin{proposition}
     	$ \mathfrak{p} $是素理想当且仅当$ R/\mathfrak{p} $是整环。
     \end{proposition}
     \begin{proof}
     考虑映射$\phi:R\rightarrow R/ \mathfrak{p}  $. 由性质\ref{prop-prime-multi}可知, $ \mathfrak{p} $是素理想当且仅当$ R/ \mathfrak{p} $是素理想, 而$ <0>\subset R/ \mathfrak{p} $, 由定义\ref{domin}可得证。
     \end{proof}
     \begin{definition}
     	$ \mathfrak{m}\subsetneq R $是$ R $的极大理想, 如果没有真理想$ \mathfrak{m}\subsetneq\mathfrak{a} $。
     \end{definition}
     \begin{proposition}
     	环$ R $是一个域当且仅当$ <0> $是一个极大理想。 $ \mathfrak{m} \subset R$是一个极大理想当且仅当$  R/ \mathfrak{m}$是一个域。
     \end{proposition}
     \begin{proof}
     设$ \mathfrak{m}\subset R $是$ R $的一个理想, $ 1\in \mathfrak{m} $, 因$ R $有逆元且$ \mathfrak{m} $有吸收性, 那么$ m=R $。
     
     考虑映射$ \phi:\mathfrak{m} \rightarrow R/ \mathfrak{m} $,$ \phi(\mathfrak{m})=0 $。 $ \mathfrak{m} $是一个极大的理想当且仅当$ <0> $在$ R/ \mathfrak{a} $中是一个极大的理想, 由上述讨论可知,$ R/ \mathfrak{m} $是一个域。
     \end{proof}
     \begin{definition}
     	给定一个环$ R $, 它的
     	     Krull dimension $ dim(R) $ 是严格递增素理想链长度的最大值
     	     \[ \text{dim}(R):=\text{sup}\{r, \text{有一个素理想链 }\mathfrak{p}_{0}\subsetneq \cdots\subsetneq\mathfrak{p}_{r}\}. \]
     \end{definition}
     \begin{example}
$      	R=\mathbb{Z} $, $ <0>\subsetneq\mathfrak{p} $, 故$\text{dim}(R)=1$;
$      	R=\mathbb{K} $, 只有一个素理想$ <0> $, 故$\text{dim}(R)=0$;
$      	R=\mathbb{Z}[x_{1},\ldots,x_{n}] $, $\text{dim}(R)=n$(暂不给出解释)。
     \end{example}
 

\end{document}
