
\documentclass[12pt,a4paper]{ctexbook} % 使用ctexbook文档类

% openany
\usepackage{graphicx} % For including images
\usepackage{tikz} %For drawing pictures
\usepackage{array}
\usepackage{tabularx} % Both are for tabular
\usepackage{subcaption}
\usepackage{hyperref} % For hyperlinks in the document
\usepackage{fancyhdr} % For custom headers and footers
\usepackage{amsthm} % For theorem environments
\usepackage{amsmath} % For additional math symbols
\usepackage{amssymb} % For additional math symbols
\usepackage{amsfonts} % For \mathbb
\usepackage{fontspec} % For font selection
\usepackage{titlesec} % For customizing section titles
\usepackage{enumitem}
\usepackage[dvipsnames]{xcolor} % 使用 xcolor 包并启用更多预定义颜色
\usepackage{wrapfig} % 用于图片环绕
\usepackage{float} % 导入 float 包
\usepackage{mathtools}
\usepackage{listings}
\lstset{
%    language=[M2]Matlab,
    basicstyle=\ttfamily\small,
    keywordstyle=\color{blue},
    commentstyle=\color{green},
    stringstyle=\color{red},
    frame=single,
    backgroundcolor=\color{gray!5},
    numbers=left
}



\setmainfont{Times New Roman}
%\newfontfamily\kaishu{Kaiti SC} % 楷体
%\newfontfamily\heiti{Heiti SC} % 黑体
%\newfontfamily\fangsong{STFangsong} % 仿宋
%\newfontfamily\songti{Songti SC} % 宋体

% 设置中文标点符号后自动添加空格
%\CJKsetecglue{,.  !?:;}

\hypersetup{
    colorlinks=true,
    linkcolor=blue,
    filecolor=magenta,
    urlcolor=cyan,
}

%\theoremstyle{upright}
\newcounter{theorem}[section]
\renewcommand{\thetheorem}{\thesection.\arabic{theorem}}

% Define a new proof environment with Fangsong font
\newenvironment{pf}[1][证明]{\par\noindent{\kaishu\textbf{#1.} }\setmainfont{STFangsong}\normalfont}{\hfill\ensuremath{\Box}\par}




\renewcommand{\thetheorem}{\thesection.\arabic{theorem}}

\newtheorem{theoreminner}[theorem]{定理}
\newtheorem*{theorem*}{定理}
\newtheorem{definition}[theorem]{定义}
\newtheorem{lemma}[theorem]{引理}
\newtheorem{proposition}[theorem]{命题}
\newtheorem{corollary}[theorem]{推论}
\newtheorem{example}[theorem]{例}
\newtheorem{remark}[theorem]{注}
\newtheorem{exercise}[theorem]{练习}
\newtheorem{conjecture}[theorem]{猜想}
\newtheorem*{remark*}{注}
\newtheorem*{conjecture*}{猜想}
\newtheorem*{example*}{例}
\newcommand{\df}[1]{\textcolor{Maroon}{#1}}

% 设置图片按照节标号
\numberwithin{figure}{section}

% Define the new problem environment
\newtheoremstyle{problemstyle} % 定义新的样式
  {3pt} % 上方空白
  {3pt} % 下方空白
  {\kaishu} % 正文字体
  {} % 缩进量
  {\bfseries} % 头部字体
  {} % 头部后标点
  { } % 头部后空格
  {\thmname{#1}\thmnumber{ #2}.\thmnote{ #3}\newline} % 头部格式,标题后换行

\theoremstyle{problemstyle}
\newtheorem{problem}{}[section]

\renewenvironment{proof}[1][证明]{\par\noindent{\kaishu\textbf{#1.} }\normalfont\fontspec{STFangsong}}{\hfill\ensuremath{\Box}\par}
% Customize section titles to add the "§" symbol and independent numbering
\titleformat{\section}[block]
  {\normalfont\Large\bfseries}
  {\S\arabic{section}}
  {1em}
  {}

% Customize subsection titles with independent numbering
\titleformat{\subsection}[block]
  {\normalfont\large\bfseries}
  {\thesection.\arabic{subsection}}
  {1em}
  {}



% Redefine section and subsection numbering to be independent
\renewcommand{\thesection}{\arabic{section}}
\renewcommand{\thesubsection}{\thesection.\arabic{subsection}}
% Define a new proof environment with STFangsong font
\renewenvironment{proof}[1][证明]{\par\noindent{\kaishu\textbf{#1.} }\normalfont\CJKfamily{zhfs}}{\hfill\ensuremath{\Box}\par}

%% Define a new remark* environment with SimHei for "注:" and Kaiti SC for the content, without using the theorem counter
%\newenvironment{remark*}[1][注记.]{\par\noindent{\textbf{#1} }\kaishu}{\par}


\hypersetup{
    colorlinks=true,
    linkcolor=blue,
    filecolor=magenta,
    urlcolor=cyan,
}

\numberwithin{equation}{section} % 按 section 编号公式


\title{{\kaishu\fontsize{48}{60}\selectfont 组~合~交~换~代~数~讲~义}} % Set the title in Kaiti font and large size
%\author{\bf }
%\date{\today}








\begin{document}

% Title page
\maketitle

\section{诺特环 $\rightarrow$ $\mathbb{K}[x_1, \ldots, x_n]$}
\subsection{诺特环的三个等价定义}
\subsubsection*{定义 1:升链条件(Ascending Chain Condition, ACC)}

一个环 $ R $ 是诺特环,当且仅当它满足理想的\emph{升链条件}。也就是说,对任意理想序列:
$$
I_1 \subseteq I_2 \subseteq I_3 \subseteq \cdots,
$$
存在正整数 $ n $,使得对所有 $ k \geq n $ 都有 $ I_k = I_n $。

\subsubsection*{定义 2:有限生成理想条件}

一个环 $ R $ 是诺特环,当且仅当它的每个理想都是有限生成的。即对任意理想 $ I \subseteq R $,存在元素 $ a_1, a_2, \dots, a_n \in R $,使得:
$$
I = \sum R a_i \coloneqq \{ \sum r_i a_i \mid r_i \in R \}
$$

\subsubsection*{定义 3:极大元条件(Maximal Element Condition)}

一个含单位元的交换环 $ R $ 是诺特环,当且仅当它的任意非空理想集合中都包含一个极大元。也就是说,设 $\mathcal{S}$ 是 $ R $ 中所有理想的非空子集,则存在一个理想 $ I \in \mathcal{S} $,使得不存在另一个理想 $ J \in \mathcal{S} $ 满足 $ I \subsetneq J $。

\subsection{诺特环的性质}

\subsubsection*{性质 1:希尔伯特基定理(Hilbert Basis Theorem)}

设环 $ R $ 是诺特环,则多项式环 $ R[x] $ 也是诺特环。

\subsubsection*{(直接推论)}
设环 $ R $ 是诺特环,则多项式环 $ R[x_1,\ldots,x_n] $ 也是诺特环。



\subsubsection*{性质 2:诺特环的商环性质}

设 $ R $ 是一个诺特环,$ I \subseteq R $ 是其一个理想,则商环 $ R/I $ 也是一个诺特环。

\subsubsection*{证明}

设 \( R \) 是诺特环,\( I \trianglelefteq R \) 是理想。考虑商环 \( R/I \) 中的任意理想升链:
\[
J_1 \subseteq J_2 \subseteq J_3 \subseteq \cdots
\]
其中 \( J_k \) 是 \( R/I \) 的理想。由商环的理想对应定理,存在 \( R \) 的理想 \( K_k \) 满足:
\[
I \subseteq K_k \quad \text{且} \quad J_k = K_k / I, \quad \forall k \geq 1.
\]
由升链包含关系 \( J_k \subseteq J_{k+1} \),可得 \( K_k \subseteq K_{k+1} \)。这是因为:
若 \( x \in K_k \),则 \( x + I \in J_k \subseteq J_{k+1} = K_{k+1}/I \),故存在 \( y \in K_{k+1} \) 使得:
\[
x + I = y + I \implies x - y \in I \subseteq K_{k+1}.
\]
因此 \( x = (x - y) + y \in K_{k+1} \),即 \( K_k \subseteq K_{k+1} \)。于是得到 \( R \) 中的理想升链:
\[
K_1 \subseteq K_2 \subseteq K_3 \subseteq \cdots
\]
因 \( R \) 是诺特环,存在 \( n \in \mathbb{Z}^+ \) 使得:
\[
K_n = K_{n+1} = K_{n+2} = \cdots
\]
从而在商环中有:
\[
J_n = K_n / I = K_{n+1} / I = J_{n+1} = \cdots
\]
故 \( R/I \) 满足理想升链条件,是诺特环。

\subsubsection*{(直接推论)}
对于多项式环 $\mathbb{K}[x_1, \ldots, x_n]$,设$I_{\triangle}$ 是一个 face ideal, 则 Stanley-Reisner 环$\mathbb{K}[\triangle]=S/I_{\triangle}$ 是一个诺特环。

\subsubsection*{性质3:Krull dimension}

\begin{enumerate}
    \item 设 $ \mathbb{K} $ 是一个域,则 $ \mathbb{K}$ 的 Krull dimension 为 1,因为最长的素理想链可以是: $$ <0>  \subsetneq \mathfrak{p}, $$
    \item 存在一个环是诺特环,但是它的 Krull dimension 不是有限的。
    \item 设 $ R $ 是一个局部诺特环(只有一个极大理想),则它的 Krull dimension是有限的。从而多元多项式环 $ \mathbb{K}[x_1, x_2, \ldots, x_n] $ 的 Krull dimension为 $ n $  (具体参考 Eisenbud-Commutative Algebra第二部分)。
    \item 对于Stanley-Reisner 环$\mathbb{K}[$\triangle$]=S/I_{\triangle}$, 
    $$(Krull)  dim(\mathbb{K}[\triangle]) = max\{|facet|\}$$
\end{enumerate}


\section{Macaulay2的使用}

\textbf{Macaulay2} 是一个开源的计算机代数系统,专为\textbf{代数几何}与\textbf{交换代数}的研究而设计。它由 Daniel R. Grayson 和 Michael E. Stillman 开发,支持对多项式环、理想、模、代数簇等数学对象进行高效计算。

\subsection{ 相关网址:}

\begin{itemize}
    \item \textbf{官网}:\url{https://macaulay2.com/}
    \item \textbf{在线版}:\url{https://www.unimelb-macaulay2.cloud.edu.au/#home}
    \item \textbf{下载网址}:\url{http://www2.macaulay2.com/Macaulay2/Downloads/}
\end{itemize}



\subsection{应用示例:}



以下是一段使用 \texttt{Macaulay2} 编写的代码,用于构造单纯复形并计算其相关代数结构,包括维数、面集合、f-向量以及 Hilbert 级数。

例1:
\begin{lstlisting}
-- 加载所需包
needsPackage "SimplicialComplexes"

-- 设置基域和多项式环
kk = QQ;
S = kk[x_1,x_2,x_3];

-- 构造单纯复形 D1
D1 = simplicialComplex {x_1*x_2, x_1*x_3, x_2*x_3};
dim D1          -- 维数
faces D1        -- 所有面
fVector D1      -- f-向量

-- 理想与商环(Stanley-Reisner环)
I_D = monomialIdeal (x_1*x_2*x_3);
SR = S/I_D;
hilbertSeries SR

\end{lstlisting}



例2:
\begin{lstlisting}
needsPackage "SimplicialComplexes"
kk = QQ;
S1 = kk[x_1..x_5];
D2 = simplicialComplex {x_1*x_2*x_3, x_2*x_4, x_3*x_4, x_5};
SR2 = S1 / ideal D2;
dim D2          -- 维数
faces D2        -- 所有面
fVector D2      -- f-向量
hilbertSeries SR2

\end{lstlisting}





 

\end{document}
