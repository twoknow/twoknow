%-------------------- 导言区 --------------------
\documentclass[12pt]{article}

% 基础字体与数学包
\usepackage[T1]{fontenc}
\usepackage{mathpazo}
\usepackage{eulervm}
\usepackage{amsmath, amsthm, amssymb}
\allowdisplaybreaks[4]

% 图形与表格
\usepackage{tikz}
\usepackage{tikz-3dplot}
\usetikzlibrary{decorations.pathreplacing, matrix}
\usepackage{nicematrix}
\usepackage[all]{xy}
\usepackage{graphicx}
\usepackage{float}
\usepackage{dcolumn}

% 其他常用包
\usepackage{bm}
\usepackage[mathscr]{eucal}
\usepackage{youngtab}
\usepackage{ytableau}
\ytableausetup{mathmode, boxsize=0.9em}
\usepackage[square]{natbib}
\usepackage{bibentry}
\usepackage{cite}

% 页面设置
\setlength{\evensidemargin}{0.3cm}
\setlength{\oddsidemargin}{0.3cm}
\parskip=6pt
\frenchspacing
\textwidth=15cm
\textheight=23cm
\parindent=16pt
\topmargin=-1.2cm

%-------------------- 定理环境 --------------------
\theoremstyle{definition}
\newtheorem{thm}{Theorem}[subsection]
\newtheorem{defi}[thm]{Definition}
\newtheorem{lemma}[thm]{Lemma}
\newtheorem{prop}[thm]{Proposition}
\newtheorem{coro}[thm]{Corollary}
\newtheorem{conj}[thm]{Conjecture}
\newtheorem*{pf}{Proof}
\newtheorem{ex}[thm]{Example}
\newtheorem{remark}[thm]{Remark}
\newtheorem{prob}{Problem}[subsection]
\numberwithin{equation}{subsection}

%-------------------- 正文开始 --------------------
\begin{document}

%-------------------- 标题与作者 --------------------
\begin{center}
    {\Large\bf Notes on Unimodality of Hypersimplex}
\end{center}
\vskip 3mm
\begin{center}
    Mingzhi Zhang
\end{center}
\vskip 3mm

%-------------------- 1. 引言 --------------------
\section{Introduction}

%---- 1.1 Hypersimplex ----
\subsection{Hypersimplex}
Fix a positive integer $n$, and let $[n]:= \{1, 2, \cdots , n\}$. To any subset $S \subseteq [n]$, we associate the indicator vector:
\[
\chi_{S} = (\chi_{S}(1), \chi_{S}(2), \ldots, \chi_{S}(n))
\]
where 
\[
\chi_{S}(i) = 
\begin{cases} 
1, & i \in S \\
0, & i \notin S
\end{cases}
\]
For $0 < k < n$, let $\binom{[n]}{k}$ be the family of all $k$-subsets of $[n]$. The \textit{hypersimplex} $\Delta_{k,n} \subseteq \mathbb{R}^n$ is the convex hull of the indicator vectors $\chi_I$ for $I \in \binom{[n]}{k}$. Equivalently,
\[
\Delta_{k,n} = \left\{ (x_1, \ldots, x_n) \mid 0 \leq x_i \leq 1,\, \sum_{i=1}^n x_i = k \right\}
\]

%---- 1.2 Stanley's Triangulations ----
\subsection{Stanley's Triangulations}
Why is the set defined by a permutation
\[
\nabla_w = \{(y_1,\dots,y_{n-1})\in [0,1]^{n-1} \mid 0 < y_{w(1)} < \cdots < y_{w(n-1)} < 1\}
\]
a simplex?

\paragraph{Step-by-Step Explanation}
\begin{enumerate}
    \item \textbf{What is a simplex?} \\
    A simplex is the simplest generalization of a triangle or tetrahedron to higher dimensions:
    \begin{itemize}
        \item 1-simplex: line segment (2 vertices)
        \item 2-simplex: triangle (3 vertices)
        \item 3-simplex: tetrahedron (4 vertices)
        \item In general, an $(n-1)$-simplex in $\mathbb{R}^{n-1}$ is the convex hull of $n$ affinely independent points.
    \end{itemize}
    Formally,
    \[
    \text{conv}(v_0, \ldots, v_d) = \left\{ \sum_{i=0}^d \lambda_i v_i \mid \lambda_i \ge 0,\, \sum_{i=0}^d \lambda_i = 1 \right\}
    \]
    \item \textbf{Connecting our definition to a simplex.} \\
    The set is defined by strict inequalities:
    \[
    0 < y_{w(1)} < \cdots < y_{w(n-1)} < 1
    \]
    The closure allows equality:
    \[
    0 \leq y_{w(1)} \leq \cdots \leq y_{w(n-1)} \leq 1
    \]
    \item \textbf{Identifying the vertices.} \\
    The vertices are:
    \[
    (0,0,\dots,0),\quad e_{w(1)},\quad e_{w(1)}+e_{w(2)},\quad \dots,\quad (1,1,\dots,1)
    \]
    where $e_i$ is the $i$-th unit vector.
    \item \textbf{Affine independence.} \\
    Each vertex introduces a new coordinate direction, so they are affinely independent.
    \item \textbf{Geometric intuition.} \\
    The region is a "slice" of the cube, forming a staircase-like path, which is a simplex.
\end{enumerate}

\paragraph{Summary}
\begin{enumerate}
    \item The set identifies $n$ vertices at cube corners.
    \item There are $n$ vertices, forming an $(n-1)$-simplex.
    \item Vertices are affinely independent.
\end{enumerate}
Thus, $\nabla_w$ is a simplex.

%---- 1.3 Volume and Ehrhart series ----
\subsection{Volume and Ehrhart series}
We are mainly interested in the volume and Ehrhart series of the characteristic polytopes. For $n$-gons up to $5$-gon:
\begin{itemize}
    \item 3-gon: 
    \begin{align*}
        \text{vol }P_{\chi}(Q_3) &= 1, \\
        \text{Ehr}_{P_{\chi}(Q_3)}(z) &= \frac{1 + 4z + z^2}{(1 - z)^3}.
    \end{align*}
    \item 4-gon: 
    \begin{align*}
        \text{vol }P_{\chi}(Q_4) &= \frac{1}{2}, \\
        \text{Ehr}_{P_{\chi}(Q_4)}(z) &= \frac{1 + 5z + 5z^2 + z^3}{(1 - z)^4}.
    \end{align*}
    \item 5-gon:
    \begin{align*}
        \text{vol }P_{\chi}(Q_5) &= \frac{5}{24}, \\
        \text{Ehr}_{P_{\chi}(Q_5)}(z) &= \frac{1 + 6z + 11z^2 + 6z^3 + z^4}{(1 - z)^5}.
    \end{align*}
\end{itemize}

From these examples, we see unimodality in the $h^*$-polynomial. Thus, we conjecture:
\begin{conj}
    The coefficients $h^*_{i}$ in the $h^*$-polynomial of the Ehrhart series of characteristic polytopes of $n$-gon form a unimodal sequence.
\end{conj}

%-------------------- 2. Characteristic Polytope of $d$-cube --------------------
\section{Characteristic Polytope of $d$-cube $\square_{d}$}

%-------------------- 参考文献 --------------------
\begin{thebibliography}{99}
    \bibitem[ABD10]{ABD10} Ardila, Federico, Carolina Benedetti, and Jeffrey Doker. "Matroid polytopes and their volumes." Discrete \& Computational Geometry 43 (2010): 841-854.
    \bibitem[AHK18]{AHK18} Adiprasito, Karim, June Huh, and Eric Katz. "Hodge theory for combinatorial geometries." Annals of Mathematics 188.2 (2018): 381-452.
    \bibitem[Ehr62]{Ehr62} Ehrhart, Eugene. "Sur les polyèdres rationnels homothétiques à n dimensions." CR Acad. Sci. Paris 254 (1962): 616.
    \bibitem[Pos09]{Pos09} Postnikov, Alexander. "Permutohedra, associahedra, and beyond." International Mathematics Research Notices 2009.6 (2009): 1026-1106.
\end{thebibliography}

\end{document}