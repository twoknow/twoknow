%-------------------- 导言区 --------------------
\documentclass[12pt]{article}

% 基础字体与数学包
\usepackage[T1]{fontenc}
\usepackage{mathpazo}
\usepackage{eulervm}
\usepackage{amsmath, amsthm, amssymb}
\allowdisplaybreaks[4]

% 图形与表格
\usepackage{tikz}
\usepackage{tikz-3dplot}
\usetikzlibrary{decorations.pathreplacing, matrix}
\usepackage{nicematrix}
\usepackage[all]{xy}
\usepackage{graphicx}
\usepackage{float}
\usepackage{dcolumn}
\usepackage{microtype} % 启用微排版以改善断行问题
\setlength{\emergencystretch}{3em} % 可选,允许额外的行间伸缩

% 其他常用包
\usepackage{bm}
\usepackage[mathscr]{eucal}
\usepackage{youngtab}
\usepackage{ytableau}
\ytableausetup{mathmode, boxsize=0.9em}
\usepackage[numbers]{natbib}
\usepackage{bibentry}
\usepackage{cite}

% 页面设置
\setlength{\evensidemargin}{0.3cm}
\setlength{\oddsidemargin}{0.3cm}
\parskip=6pt
\frenchspacing
\textwidth=15cm
\textheight=23cm
\parindent=16pt
\topmargin=-1.2cm

%-------------------- 定理环境 --------------------
\theoremstyle{definition}
\newtheorem{thm}{Theorem}[subsection]
\newtheorem{defi}[thm]{Definition}
\newtheorem{lemma}[thm]{Lemma}
\newtheorem{prop}[thm]{Proposition}
\newtheorem{coro}[thm]{Corollary}
\newtheorem{conj}[thm]{Conjecture}
\newtheorem*{pf}{Proof}
\newtheorem{ex}[thm]{Example}
\newtheorem{remark}[thm]{Remark}
\newtheorem{prob}{Problem}[subsection]
\numberwithin{equation}{subsection}

%-------------------- 正文开始 --------------------
\begin{document}
\sloppy
%-------------------- 标题与作者 --------------------
\begin{center}
    {\Large\bf Notes on Stanley-Reisner Ring}
\end{center}
\vskip 3mm
\begin{center}
    Mingzhi Zhang
\end{center}
\vskip 3mm

\section*{Part I: Foundational Algebraic Concepts}
% 第一部分:基础代数概念

The study of Stanley-Reisner rings is deeply rooted in commutative algebra. Understanding basic algebraic structures such as rings, ideals, quotient rings, Noetherian rings, and Krull dimension is essential before exploring their combinatorial counterparts.

\subsection{Rings, Ideals, and Quotient Rings}
% 子部分:环、理想与商环

The algebraic landscape of this field is built upon the concept of a ring.

\paragraph{Rings}
% 段落:环
A \textbf{ring} $R$ is a set equipped with two binary operations, typically called addition ($+$) and multiplication ($\cdot$), satisfying specific axioms. Specifically:

\begin{enumerate}
  \item $(R,+)$ is an abelian group:
    \begin{itemize}
      \item Addition is associative: $(a+b)+c=a+(b+c)$ for all $a,b,c\in R$.
      \item Addition is commutative: $a+b=b+a$ for all $a,b\in R$.
      \item There exists an additive identity $0\in R$ such that $a+0=a$ for all $a\in R$.
      \item For each $a\in R$, there exists an additive inverse $-a\in R$ such that $a+(-a)=0$.
    \end{itemize}
  \item $(R,\cdot)$ is a monoid (though some definitions vary, rings in this context are generally assumed to have a multiplicative identity):
    \begin{itemize}
      \item Multiplication is associative: $(a\cdot b)\cdot c=a\cdot(b\cdot c)$ for all $a,b,c\in R$.
      \item There exists a multiplicative identity $1\in R$ such that $a\cdot 1=1\cdot a=a$ for all $a\in R$.
    \end{itemize}
  \item Multiplication is distributive over addition:
    \begin{itemize}
      \item $a\cdot(b+c)=(a\cdot b)+(a\cdot c)$ (left distributivity) for all $a,b,c\in R$.
      \item $(b+c)\cdot a=(b\cdot a)+(c\cdot a)$ (right distributivity) for all $a,b,c\in R$.
    \end{itemize}
\end{enumerate}

A ring is \textbf{commutative} if $a\cdot b=b\cdot a$ for all $a,b\in R$. Polynomial rings over a field, such as $k[x_1,\ldots,x_n]$, are primary examples of commutative rings with identity and are central to Stanley-Reisner theory.

\paragraph{Ideals}
% 段落:理想
An \textbf{ideal} $I$ of a ring $R$ is a special subset that generalizes concepts like "multiples of an integer $n$" in the ring of integers $\mathbb{Z}$, or "polynomials vanishing on a specific geometric set" in a polynomial ring. Formally, a non-empty subset $I\subseteq R$ is a \textbf{left ideal} if:

\begin{enumerate}
  \item For all $a,b\in I$, $a-b\in I$ ($I$ is an additive subgroup).
  \item For all $r\in R$ and $a\in I$, $ra\in I$ (absorption from the left).
\end{enumerate}

A \textbf{right ideal} is defined analogously with $ar\in I$. A \textbf{two-sided ideal} (or simply an ideal, especially in commutative rings) is both a left and a right ideal. An ideal $I$ is \textbf{proper} if $I\neq R$. The zero ideal $\{0\}$ and $R$ itself are always ideals.

\paragraph{Quotient Rings}
% 段落:商环
Given a ring $R$ and a two-sided ideal $I$, the \textbf{quotient ring} (or factor ring) $R/I$ is formed by the set of cosets $\{a+I\mid a\in R\}$. Addition and multiplication in $R/I$ are defined as:

\begin{itemize}
  \item $(a+I)+(b+I)=(a+b)+I$
  \item $(a+I)\cdot(b+I)=(a\cdot b)+I$
\end{itemize}

These operations are well-defined precisely because $I$ is a two-sided ideal. The quotient ring $R/I$ inherits many properties from $R$; for instance, if $R$ is commutative, then $R/I$ is commutative. The construction of quotient rings is a fundamental technique in algebra, allowing for the "simplification" of rings by treating all elements of an ideal $I$ as equivalent to zero. The properties of $R/I$ often reveal crucial information about the ideal $I$ itself.

\paragraph{Fields and Integral Domains}
A commutative ring $R$ with identity $1 \neq 0$ is a \textbf{field} if every non-zero element has a multiplicative inverse. Examples include $\mathbb{Q}$, $\mathbb{R}$, $\mathbb{C}$, and $\mathbb{Z}_p$ for prime $p$.
An element $a \in R$ is a \textbf{zero divisor} if there exists a non-zero $b \in R$ such that $ab=0$. A commutative ring $R$ with identity $1 \neq 0$ is an \textbf{integral domain} if it has no zero divisors. Examples include $\mathbb{Z}$ and polynomial rings $k[x_1, \ldots, x_n]$ over a field $k$.

\paragraph{Prime Ideals} 

A proper ideal $P$ in a commutative ring $R$ is a \textbf{prime ideal} if for any $a,b \in R$, whenever $ab \in P$, then $a \in P$ or $b \in P$. This definition is a direct generalization of prime numbers in Z: an integer p>1 is prime if and only if the ideal pZ is a prime ideal in Z. For example, if $ab \in 3Z$, then 3 divides ab, implying 3 divides a or 3 divides b; thus $a \in 3Z$ or $b \in 3Z$.   

\paragraph{Maximal Ideals} 

A proper ideal $M$ in a ring $R$ is a \textbf{maximal ideal} if there is no other proper ideal $J$ of $R$ such that $M \subsetneq J \subsetneq R$. In other words, $M$ is a maximal element in the set of proper ideals of $R$, partially ordered by inclusion.   

Several important theorems connect these types of ideals with the structure of their corresponding quotient rings, especially in commutative rings with identity:
\begin{itemize}
\item An ideal $I$ in a commutative ring $R$ with identity is \textbf{maximal} if and only if the quotient ring $R/I$ is a \textbf{field}. A field is a commutative ring where every non-zero element has a multiplicative inverse. The intuition is that if $M$ is maximal, $R/M$ cannot be "simplified" further without collapsing to the zero ring, a characteristic property of fields.
\item An ideal $P$ in a commutative ring $R$ with identity is \textbf{prime} if and only if the quotient ring $R/P$ is an \textbf{integral domain}. An integral domain is a commutative ring with identity that has no zero divisors (i.e., if $xy=0$, then $x=0$ or $y=0$). The condition $ab \in P \implies a \in P$ or $b \in P$ translates directly to $(a+P)(b+P)=0+P \implies a+P=0+P$ or $b+P=0+P$ in $R/P$.
\item As a direct consequence, every \textbf{maximal ideal} in a commutative ring with identity is also a \textbf{prime ideal}, because every field is an integral domain.
\end{itemize}   

\paragraph{Examples}
\begin{itemize}
\item In $\mathbb{Z}$, ideals are of the form $n\mathbb{Z}$. $p\mathbb{Z}$ is maximal (and prime) if $p$ is a prime number, because $\mathbb{Z}/p\mathbb{Z}$ is a field. $n\mathbb{Z}$ is prime if and only if $n$ is prime (or $n=0$). If $n$ is composite (e.g., $n=4$), $n\mathbb{Z}$ is not prime (e.g., $2\cdot 2=4 \in 4\mathbb{Z}$ but $2 \notin 4\mathbb{Z}$), and $\mathbb{Z}/n\mathbb{Z}$ is not an integral domain (e.g., in $\mathbb{Z}/4\mathbb{Z}$,  
4Z), and Z/nZ is not an integral domain (e.g., in Z/4Z).
\item In the polynomial ring $K[x,y]$ over a field $K$, the ideal $\langle y \rangle$ (polynomials with $y$ as a factor) is prime because $K[x,y]/\langle y \rangle \cong K[x]$, which is an integral domain. However, $\langle y \rangle$ is not maximal because $K[x]$ is not a field. For instance, $\langle y \rangle \subsetneq \langle x,y \rangle \subsetneq K[x,y]$, and $K[x,y]/\langle x,y \rangle \cong K$, which is a field, so $\langle x,y \rangle$ is maximal.
\end{itemize}   

\subsection{Noetherian Rings}

Noetherian rings, named after Emmy Noether, are a class of rings with particularly well-behaved ideal structures. They are fundamental in commutative algebra and algebraic geometry.


\paragraph{Definition} 
A ring $R$ is \textbf{left-Noetherian} if it satisfies the \textbf{ascending chain condition (ACC)} on its left ideals. This means that for any sequence of left ideals
\[
I_1 \subseteq I_2 \subseteq I_3 \subseteq \cdots
\]
there exists an integer $N$ such that
\[
I_N = I_{N+1} = I_{N+2} = \cdots
\]
i.e., the chain stabilizes. Similarly, a ring is \textbf{right-Noetherian} if it satisfies ACC on right ideals. A ring is simply called \textbf{Noetherian} if it is both left- and right-Noetherian. For commutative rings, the distinction between left, right, and two-sided ACC is not necessary, as they are all equivalent.


\paragraph{Equivalent Conditions} 
For a ring $R$, the following are equivalent to $R$ being left-Noetherian:
\begin{enumerate}
\item Every left ideal $I$ in $R$ is \textbf{finitely generated}. This means there exist elements $a_1, \ldots, a_k \in I$ such that $I = Ra_1 + \cdots + Ra_k = \{r_1a_1 + \cdots + r_ka_k \mid r_j \in R\}$. The ability to describe any ideal with a finite set of generators is a powerful property.
\item Every non-empty set of left ideals of $R$, partially ordered by inclusion, has a \textbf{maximal element}.
\end{enumerate}

\begin{proof}
We prove the equivalence by showing (1) $\Rightarrow$ (2) $\Rightarrow$ (3) $\Rightarrow$ (1).

\textbf{(1) $\Rightarrow$ (2):} Assume $R$ satisfies the ACC on left ideals. Let $I$ be an arbitrary left ideal of $R$. We want to show that $I$ is finitely generated.
Suppose, for the sake of contradiction, that $I$ is not finitely generated.
Then we can construct an infinite strictly ascending chain of finitely generated left sub-ideals of $I$:
Pick an element $a_1 \in I$. Let $I_1 = Ra_1$. $I_1$ is finitely generated and $I_1 \subseteq I$.
Since $I$ is not finitely generated by $\{a_1\}$, $I_1 \subsetneq I$. So there exists an element $a_2 \in I \setminus I_1$. Let $I_2 = Ra_1 + Ra_2 = I_1 + Ra_2$. $I_2$ is finitely generated, $I_1 \subsetneq I_2$, and $I_2 \subseteq I$.
Continuing this process, we obtain an infinite strictly ascending chain of left ideals:
$I_1 \subsetneq I_2 \subsetneq I_3 \subsetneq \cdots$
where each $I_k$ is a finitely generated left ideal contained in $I$.
This infinite strictly ascending chain contradicts the assumption that $R$ satisfies the ACC.
Therefore, our initial supposition that $I$ is not finitely generated must be false.
Thus, every left ideal $I$ in $R$ must be finitely generated.

\textbf{(2) $\Rightarrow$ (3):} Assume every left ideal $I$ in $R$ is finitely generated. Let $\mathcal{S}$ be a non-empty set of left ideals of $R$. We want to show that $\mathcal{S}$ has a maximal element. Suppose for contradiction that $\mathcal{S}$ has no maximal element. Then for any ideal $I \in \mathcal{S}$, there exists an ideal $J \in \mathcal{S}$ such that $I \subsetneq J$. We can construct an infinite strictly ascending chain of ideals:
$I_1 \subsetneq I_2 \subsetneq I_3 \subsetneq \cdots$
where each $I_k \in \mathcal{S}$. Let $I = \bigcup_{k=1}^{\infty} I_k$. Then $I$ is a left ideal of $R$. Since $I$ is a left ideal, by assumption (2), $I$ must be finitely generated. Let $I = \langle a_1, \ldots, a_m \rangle$ for some $a_1, \ldots, a_m \in I$. Since each $a_j \in I = \bigcup_{k=1}^{\infty} I_k$, each $a_j$ must belong to some $I_{k_j}$. Let $N = \max\{k_1, \ldots, k_m\}$. Then all $a_j \in I_N$. This implies $I \subseteq I_N$. Since $I_N \subseteq I$ by construction, we have $I = I_N$. But this contradicts the fact that the chain $I_N \subsetneq I_{N+1}$ is strictly ascending. Therefore, our assumption that $\mathcal{S}$ has no maximal element must be false, and thus $\mathcal{S}$ must have a maximal element. Thus, (2) implies (3).

\textbf{(3) $\Rightarrow$ (1):} Assume every non-empty set of left ideals of $R$ has a maximal element. Let $I$ be any left ideal of $R$. We want to show that $I$ is finitely generated. Consider the set $\mathcal{F}$ of all finitely generated left ideals contained in $I$. $\mathcal{F}$ is non-empty since $\{0\} \in \mathcal{F}$. By assumption (3), $\mathcal{F}$ has a maximal element, say $J$. We claim that $J=I$. Suppose for contradiction that $J \subsetneq I$. Then there exists an element $x \in I$ such that $x \notin J$. Consider the ideal $J' = J + Rx$. Since $J$ is finitely generated and $Rx$ is finitely generated (by $x$), $J'$ is also finitely generated. Also, $J \subsetneq J' \subseteq I$. This contradicts the maximality of $J$ in $\mathcal{F}$. Therefore, $J$ must be equal to $I$, which means $I$ is finitely generated. Thus, (3) implies (1).

Combining these implications, we conclude that the three conditions are equivalent.
\end{proof}

\paragraph{Primary Ideals}
A proper ideal $Q$ in a commutative ring $R$ is a \textbf{primary ideal} if for any $a,b \in R$, whenever $ab \in Q$, then either $a \in Q$ or $b^n \in Q$ for some positive integer $n$. This generalizes the concept of prime power ideals in $\mathbb{Z}$ (e.g., $p^n\mathbb{Z}$).

\paragraph{Key Properties of Noetherian Rings}
The Noetherian property is preserved under several important algebraic constructions, making it a robust and widely applicable concept:
\begin{itemize}
\item \textbf{Hilbert's Basis Theorem:} If $R$ is a (left/right) Noetherian ring, then the polynomial ring $R[X]$ is also (left/right) Noetherian.\\
By induction, $R[X_1, x_2, \cdots, x_n]$ is Noetherian. Also, the power series ring $R[[X]]$ is Noetherian if $R$ is.\\
This theorem, first proven by David Hilbert in 1890 in his paper "Über die Theorie der algebraischen Formen" (On the Theory of Algebraic Forms), was a groundbreaking result in invariant theory. Hilbert's original formulation showed that if a field $k$ is Noetherian (which all fields are), then any ideal in the polynomial ring $k[x_1, \ldots, x_n]$ is finitely generated. The modern statement generalizes this to any Noetherian ring $R$, asserting that $R[x_1, \ldots, x_n]$ is also Noetherian. This theorem is of paramount importance because it implies that polynomial rings over fields (which are Noetherian) are themselves Noetherian, providing a finite basis for all their ideals. Stanley-Reisner rings are quotients of such polynomial rings.
\item \textbf{Quotients:} If $R$ is a Noetherian ring and $I$ is a two-sided ideal, then the quotient ring $R/I$ is also Noetherian.
\begin{proof}
Let $R$ be a Noetherian ring and $I$ be a two-sided ideal of $R$. We want to show that $R/I$ is Noetherian. We will use the ascending chain condition (ACC) on ideals.
Let $J_1 \subseteq J_2 \subseteq J_3 \subseteq \cdots$ be an ascending chain of ideals in $R/I$.
For each ideal $J_k$ in $R/I$, consider its preimage under the canonical projection map $\pi: R \to R/I$, defined by $\pi(r) = r+I$. Let $A_k = \pi^{-1}(J_k) = \{r \in R \mid r+I \in J_k\}$.
Each $A_k$ is an ideal in $R$. To see this:
\begin{itemize}
    \item $0 \in A_k$ since $0+I \in J_k$.
    \item If $x, y \in A_k$, then $x+I \in J_k$ and $y+I \in J_k$. Since $J_k$ is an ideal, $(x+I)-(y+I) = (x-y)+I \in J_k$, so $x-y \in A_k$.
    \item If $x \in A_k$ and $r \in R$, then $x+I \in J_k$. Since $J_k$ is an ideal, $(r+I)(x+I) = (rx)+I \in J_k$, so $rx \in A_k$. Similarly, $xr \in A_k$.
\end{itemize}
Furthermore, since $J_k \subseteq J_{k+1}$, it follows that $A_k \subseteq A_{k+1}$. Thus, we have an ascending chain of ideals in $R$:
$A_1 \subseteq A_2 \subseteq A_3 \subseteq \cdots$
Since $R$ is Noetherian, this chain must stabilize. Therefore, there exists an integer $N$ such that $A_N = A_{N+1} = A_{N+2} = \cdots$.
Now, we need to show that this implies the chain $J_1 \subseteq J_2 \subseteq J_3 \subseteq \cdots$ also stabilizes.
For any $k \ge N$, we have $A_k = A_N$.
Since $\pi$ is a surjective homomorphism, $\pi(A_k) = J_k$.
Thus, for $k \ge N$, $J_k = \pi(A_k) = \pi(A_N) = J_N$.
This shows that the chain of ideals in $R/I$ stabilizes at $J_N$.
Therefore, $R/I$ satisfies the ascending chain condition on ideals, which means $R/I$ is a Noetherian ring.
\end{proof}
\item \textbf{Localizations:} Every localization of a commutative Noetherian ring is Noetherian.
\item \textbf{Finite Generation:} In a commutative Noetherian ring, every ideal being finitely generated implies a certain ``tameness.''\\
Many algorithms and structural theorems rely on this finite generation property.
\item \textbf{Minimal Prime Ideals:} A commutative Noetherian ring has only a finite number of minimal prime ideals.
\item \textbf{Primary Decomposition:} In a commutative Noetherian ring, every ideal has a primary decomposition, meaning it can be written as a finite intersection of primary ideals. This is analogous to prime factorization of integers.
\end{itemize}
The significance of Noetherian rings lies in their "finite" nature concerning ideals. This finiteness condition ensures that many algebraic procedures terminate and that ideals have a manageable structure. Most rings encountered in algebraic geometry, including Stanley-Reisner rings (as they are quotients of polynomial rings over fields), are Noetherian.   


\subsection{Krull Dimension}
The \textbf{Krull dimension} of a ring is a measure of the "size" or "complexity" of its ideal structure, particularly in terms of chains of prime ideals. It is a fundamental concept in commutative algebra and algebraic geometry.
\paragraph{Definition}
The \textbf{Krull dimension} of a commutative ring $R$, denoted $\dim(R)$, is defined as the supremum of the lengths of all chains of prime ideals in $R$. A chain of prime ideals is a sequence
\[
P_0 \subsetneq P_1 \subsetneq P_2 \subsetneq \cdots \subsetneq P_n
\]
where each $P_i$ is a prime ideal of $R$, and the length of this chain is $n$. The Krull dimension can be finite or infinite, depending on the ring.

\paragraph{Height of a Prime Ideal}
The \textbf{height} of a prime ideal $P$ in a commutative ring $R$, denoted $\mathrm{ht}(P)$, is the supremum of the lengths of all chains of prime ideals $P_0 \subsetneq P_1 \subsetneq \cdots \subsetneq P_n = P$ that end at $P$. The Krull dimension of a ring $R$ is then the supremum of the heights of all prime ideals in $R$.
\paragraph{Properties}
\begin{itemize}
\item \textbf{Noetherian Rings:} For Noetherian rings, the Krull dimension is well-defined and finite. The dimension can be computed using the ascending chain condition on prime ideals.
\item \textbf{Zero-Dimensional Rings:} A ring is zero-dimensional if its Krull dimension is zero, meaning every prime ideal is maximal. This implies that the ring has a finite number of maximal ideals.
\item \textbf{Dimension of Polynomial Rings:} If $R$ is a Noetherian ring, then the Krull dimension of the polynomial ring $R[x]$ is $\dim(R) + 1$. This property is crucial for understanding the dimensions of varieties defined by polynomial equations.
\item \textbf{Dimension of Quotient Rings:} If $R$ is a Noetherian ring and $I$ is a prime ideal, then $\dim(R/I) = \dim(R) - 1$. This property helps in analyzing the dimensions of varieties defined by quotient rings.
\item \textbf{Applications in Algebraic Geometry:} The Krull dimension is closely related to the geometric notion of dimension. For example, the Krull dimension of the coordinate ring of an algebraic variety corresponds to the topological dimension of the variety.
\item \textbf{Chain Conditions:} The Krull dimension is related to the ascending and descending chain conditions on prime ideals. A ring with finite Krull dimension satisfies the ascending chain condition on prime ideals, while a ring with finite Krull dimension also satisfies the descending chain condition on prime ideals.
\end{itemize}
\paragraph{Examples}
\begin{itemize}
\item The ring of integers $\mathbb{Z}$ has Krull dimension 1, as the only prime ideals are $(0)$ and the maximal ideals $(p)$ for prime numbers $p$.
\item The polynomial ring $k[x,y]$ over a field $k$ has Krull dimension 2, as it contains chains of prime ideals like $(0) \subsetneq (x) \subsetneq (x,y)$.
\item The ring of continuous functions on a compact space has Krull dimension equal to the topological dimension of the space.
\item The ring of formal power series $k[[x]]$ over a field $k$ has Krull dimension 1, as it has a unique maximal ideal $(x)$.
\end{itemize}

\section*{Part II: The Stanley-Reisner Ring}
% 第二部分:Stanley-Reisner环

The Stanley-Reisner ring is a fundamental construction in combinatorial commutative algebra, linking combinatorial properties of simplicial complexes to algebraic properties of rings. It plays a crucial role in the study of combinatorial invariants and has applications in algebraic geometry, topology, and combinatorics.

\subsection{Simplicial Complexes}
The combinatorial foundation of Stanley-Reisner theory is the concept of a simplicial complex.

\paragraph{Definition}
A \textbf{simplicial complex} $\Delta$ on a finite set of vertices $V = \{v_1, v_2, \ldots, v_n\}$ is a collection of subsets of $V$ such that:
\begin{enumerate}
    \item For each $v \in V$, $\{v\} \in \Delta$. (Every vertex is a face.)
    \item If $F \in \Delta$ and $G \subseteq F$, then $G \in \Delta$. (Every subset of a face is also a face.)
\end{enumerate}
The elements of $\Delta$ are called \textbf{faces} or \textbf{simplices}. If a face $F$ has $k+1$ elements, it is called a $k$-dimensional face or $k$-simplex. The dimension of a face is one less than its cardinality. The dimension of a simplicial complex $\Delta$, denoted $\dim(\Delta)$, is the maximum dimension of its faces.

\paragraph{Maximal Faces and Pure Complexes}
A face $F \in \Delta$ is a \textbf{maximal face} (or \textbf{facet}) if it is not a proper subset of any other face in $\Delta$.
A simplicial complex $\Delta$ is \textbf{pure} if all its maximal faces have the same dimension.

\paragraph{Missing Faces}
In the context of the Stanley-Reisner ideal, a subset $S \subseteq V$ is a \textbf{missing face} (or \textbf{non-face}) of $\Delta$ if $S \notin \Delta$. These missing faces are precisely what generate the Stanley-Reisner ideal.

\subsection{Stanley-Reisner Rings}
Given a simplicial complex $\Delta$ on the vertex set $[n] = \{1, 2, \ldots, n\}$, the \textbf{Stanley-Reisner ring} (or face ring) of $\Delta$, denoted $k[\Delta]$, is defined as follows:
\begin{itemize}
\item Let $k$ be a field. The \textbf{Stanley-Reisner ideal} $I_\Delta$ is generated by the monomials $x_S = x_{i_1} x_{i_2} \cdots x_{i_k}$, where $S = \{i_1, i_2, \ldots, i_k\}$ is a subset of $[n]$ that is not a face of $\Delta$ (i.e., $S$ is not contained in any simplex of $\Delta$).
\item The \textbf{Stanley-Reisner ring} is then defined as the quotient ring:
\[
k[\Delta] = k[x_1, x_2, \ldots, x_n]/I_\Delta
\]
where $k[x_1, x_2, \ldots, x_n]$ is the polynomial ring in $n$ variables over the field $k$.
\end{itemize}
\paragraph{Properties}
The Stanley-Reisner ring $k[\Delta]$ has several important properties:
\begin{itemize}
\item \textbf{Graded Structure:} The ring $k[\Delta]$ is graded by the total degree of the monomials. The degree of a monomial $x_S$ is equal to the size of the set $S$. This grading reflects the combinatorial structure of the simplicial complex.
\item \textbf{Hilbert Series:} The Hilbert series of $k[\Delta]$ encodes information about the dimensions of the graded components of the ring. It is defined as:
\[
H_{k[\Delta]}(t) = \sum_{d=0}^{\infty} \dim_k (k[\Delta]_d) t^d
\]
where $k[\Delta]_d$ is the degree $d$ component of the ring. The Hilbert series can be computed using combinatorial techniques and is closely related to the f-vector and h-vector of the simplicial complex.

\paragraph{f-vector and h-vector}
For a simplicial complex $\Delta$ of dimension $d-1$, let $f_i$ be the number of $i$-dimensional faces of $\Delta$ for $i=0, \ldots, d-1$. The sequence $(f_0, f_1, \ldots, f_{d-1})$ is called the \textbf{f-vector} of $\Delta$. The \textbf{f-polynomial} is $f(t) = \sum_{i=0}^{d-1} f_i t^i$.
The \textbf{h-vector} $(h_0, h_1, \ldots, h_d)$ is related to the f-vector by the equation:
\[
\sum_{i=0}^d h_i t^{d-i} = \sum_{i=0}^d f_{i-1} (t-1)^{d-i}
\]
where $f_{-1}=1$ (representing the empty face). The h-vector is a fundamental invariant in combinatorial commutative algebra, often revealing deeper properties of the simplicial complex and its Stanley-Reisner ring.

\item \textbf{Cohen-Macaulay Property:} The Stanley-Reisner ring is Cohen-Macaulay if and only if the simplicial complex $\Delta$ is pure (all maximal faces have the same dimension) and has no missing faces. This property is crucial for understanding the algebraic structure of the ring.
\item \textbf{Reisner's Theorem:} This theorem states that the Stanley-Reisner ring $k[\Delta]$ is Cohen-Macaulay if and only if the simplicial complex $\Delta$ is pure and has no missing faces. This result connects the combinatorial properties of the complex to the algebraic properties of the ring.
\item \textbf{Face Poset:} The \textbf{face poset} of a simplicial complex $\Delta$, denoted $\mathcal{F}(\Delta)$, is the set of all faces of $\Delta$ ordered by inclusion. This poset is a fundamental combinatorial object that encodes the entire structure of the simplicial complex.

\item \textbf{Hochster's Formula:} This formula relates the Betti numbers of the Stanley-Reisner ring to the combinatorial structure of the simplicial complex. It provides a way to compute the Betti numbers using the face poset of $\Delta$.

\paragraph{Betti Numbers}
The \textbf{Betti numbers} $\beta_{i,j}(R)$ of a graded $k$-algebra $R$ measure the complexity of its minimal free resolution. Specifically, $\beta_{i,j}(R)$ is the number of generators of degree $j$ in the $i$-th free module of the minimal free resolution of $R$ as a module over a polynomial ring. They provide deep insights into the homological properties of the ring.

\paragraph{Artinian Rings}
A ring $R$ is called \textbf{Artinian} if it satisfies the descending chain condition (DCC) on its ideals. This means that for any sequence of ideals $I_1 \supseteq I_2 \supseteq I_3 \supseteq \cdots$, there exists an integer $N$ such that $I_N = I_{N+1} = I_{N+2} = \cdots$. For commutative rings, a ring is Artinian if and only if it is Noetherian and has Krull dimension zero.

\item \textbf{Artinian Property:} The Stanley-Reisner ring is Artinian if and only if the simplicial complex $\Delta$ is finite and has no missing faces. This property is important for understanding the structure of the ring in terms of its minimal free resolution.

\paragraph{Minimal Free Resolution}
For a graded $k$-algebra $R$ (such as $k[\Delta]$) that is a quotient of a polynomial ring $S = k[x_1, \ldots, x_n]$, a \textbf{minimal free resolution} of $R$ as an $S$-module is an exact sequence of graded free $S$-modules:
\[
\cdots \to F_2 \to F_1 \to F_0 \to R \to 0
\]
where each $F_i = \bigoplus_j S(-j)^{\beta_{i,j}}$ and the maps are chosen such that the images are contained in the maximal ideal times the previous module. The minimal free resolution provides a fundamental way to understand the algebraic structure of $R$, and its Betti numbers $\beta_{i,j}$ are key invariants.
\item \textbf{Lefschetz Properties:} The Stanley-Reisner ring satisfies various Lefschetz properties, which have implications for the unimodality and log-concavity of the h-vector. These properties are crucial for understanding the combinatorial structure of the simplicial complex.

\paragraph{Unimodality and Log-concavity}
A sequence of non-negative real numbers $(a_0, a_1, \ldots, a_d)$ is \textbf{unimodal} if there exists an index $k$ such that $a_0 \le a_1 \le \cdots \le a_k \ge a_{k+1} \ge \cdots \ge a_d$. It is \textbf{log-concave} if $a_i^2 \ge a_{i-1}a_{i+1}$ for all $i=1, \ldots, d-1$. Log-concavity implies unimodality. These properties are significant in combinatorics, often arising from underlying algebraic or geometric structures.

\item \textbf{Partition Complex:} The partition complex, introduced by Adiprasito, provides a combinatorial framework for studying the homology of simplicial complexes and their Stanley-Reisner rings. It simplifies proofs of Reisner's theorem and Schenzel's formula.
\end{itemize}
\paragraph{Examples}
\begin{itemize}
\item \textbf{Simplicial Complexes:} For a simplicial complex $\Delta$ with vertices $\{1, 2, 3\}$ and faces $\{\{1\}, \{2\}, \{3\}, \{1, 2\}\}$, the Stanley-Reisner ideal is generated by $x_3$, and the Stanley-Reisner ring is $k[x_1, x_2]/(x_3)$.
\item \textbf{Empty Complex:} For the empty simplicial complex, the Stanley-Reisner ring is simply $k[x_1, x_2, \ldots, x_n]$, as there are no missing faces.
\item \textbf{Full Complex:} For the full simplicial complex on $n$ vertices, the Stanley-Reisner ring is $k[x_1, x_2, \ldots, x_n]$, as there are no missing faces.
\item \textbf{Graph Ideals:} The Stanley-Reisner ring of a graph can be constructed by considering the simplicial complex formed by its edges. For example, for a complete graph $K_n$, the Stanley-Reisner ring is $k[x_1, x_2, \ldots, x_n]$.
\item \textbf{Partition Complex:} The partition complex of a set provides a combinatorial framework for studying the homology of simplicial complexes and their Stanley-Reisner rings. It simplifies proofs of Reisner's theorem and Schenzel's formula.
\end{itemize}

\section*{Summary}
% 总结
The journey from Stanley's pioneering work to the recent breakthroughs by Adiprasito reveals a significant evolution in techniques while underscoring the enduring importance of core concepts.

\paragraph{Evolution of Techniques}
\begin{itemize}
\item \textbf{Stanley's Era (1970s-1990s):} This period established the foundational framework. Key tools included Stanley-Reisner rings, the Cohen-Macaulay property (via Reisner's theorem), Hochster's formula for Betti numbers, and the systematic use of Hilbert functions and h-vectors. The primary strategy was to translate combinatorial problems (like face counting or shellability) into algebraic questions about Stanley-Reisner rings and then leverage their homological and ring-theoretic properties. This approach successfully resolved the UBC for spheres.

\paragraph{Upper Bound Conjecture (UBC)}
The \textbf{Upper Bound Conjecture} (now a theorem by Stanley) states that among all simplicial $(d-1)$-spheres with $n$ vertices, the boundary complex of a cyclic polytope with $n$ vertices has the maximum number of faces of each dimension. This conjecture was a major open problem in combinatorial geometry.

\paragraph{Shellability}
A simplicial complex $\Delta$ is \textbf{shellable} if its facets (maximal faces) can be ordered $F_1, F_2, \ldots, F_m$ such that for each $j \in \{2, \ldots, m\}$, the intersection of $F_j$ with the union of the preceding facets $\bigcup_{i=1}^{j-1} F_i$ is a pure $(d-2)$-dimensional simplicial complex (where $d-1$ is the dimension of $F_j$). Shellability is a strong combinatorial property that implies Cohen-Macaulayness.

\item \textbf{Adiprasito's Era (2010s-Present):} This era builds upon these foundations but introduces novel, often more direct, combinatorial-topological tools to establish deep algebraic properties (like Lefschetz theorems and Poincaré duality) for face rings and their Artinian reductions. These new methods sometimes circumvent the need for more heavyweight abstract algebraic machinery (like local cohomology in its full generality for certain proofs) or extend results to settings where traditional "positivity" arguments (e.g., from Kähler geometry) do not directly apply.
\end{itemize}


\end{document}