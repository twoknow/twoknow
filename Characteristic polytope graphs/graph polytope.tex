\documentclass[12pt]{article}

\usepackage[T1]{fontenc}
\usepackage{mathpazo}
\usepackage{eulervm}

\usepackage{amsmath}
\usepackage{amsthm}
\allowdisplaybreaks[4]
\usepackage{amssymb}

\usepackage{tikz}
\usepackage{tikz-3dplot}
\usetikzlibrary{decorations.pathreplacing, matrix}
\usepackage{nicematrix}
\usetikzlibrary{decorations.pathreplacing}
\usepackage{xy}  % Consider loading xy-pic only if needed

\usepackage{bm}
\usepackage{dcolumn} %  This package is for aligning numbers in columns, useful for tables.
\usepackage[mathscr]{eucal} % This package provides script fonts for mathematical calligraphic symbols.
\usepackage{float}  % For float placement control (e.g., [H] for "exactly here")
\usepackage{graphicx} % For including images

\usepackage{times} % This package is deprecated. Consider using a different font package if needed.
\usepackage[pdftex]{epstopdf} % 为 epstopdf 指定 pdftex 驱动程序

\usepackage[square,numbers]{natbib} % 修改为数字引用样式,解决与作者-年份引用样式的不兼容问题
\usepackage{bibentry}
\usepackage{cite} % For grouped citations like (1, 2, 3)


\usepackage{youngtab}  % For Young tableaux
\usepackage{ytableau}
\ytableausetup{mathmode, boxsize=0.9em}

\setlength{\evensidemargin}{0.3cm} % Consider unifying these lengths
\setlength{\oddsidemargin}{0.3cm}  % Consider unifying these lengths
\parskip=6pt
\frenchspacing
\textwidth=15cm
\textheight=23cm
\parindent=16pt
\topmargin=-1.2cm

% Theorem styles
\theoremstyle{definition}
\newtheorem{thm}{Theorem}[subsection]
\newtheorem{defi}[thm]{Definition}
\newtheorem{lemma}[thm]{Lemma}
\newtheorem{prop}[thm]{Proposition}
\newtheorem{coro}[thm]{Corollary}
\newtheorem{conj}[thm]{Conjecture}
\newtheorem*{pf}{Proof}

\newtheorem{ex}[thm]{Example}
\newtheorem{remark}[thm]{Remark}

\newtheorem{prob}{Problem}[subsection]

\numberwithin{equation}{subsection}

%\usepackage{hyperref}  % Load hyperref LAST (important!)


%-------------------------------------------------------------
\begin{document}


\begin{center}
{\Large\bf 
Notes on Characteristic Polytope
%and 
%Combinatorial Classes for \\[10pt]
%Restrictions of a Hyperplane Arrangement
}\\ [7pt]
\end{center}

\vskip 3mm

\begin{center}
Mingzhi Zhang
\end{center}

\vskip 3mm

\section{Characteristic Polytope}

\subsection{Characteristic vectors}
Fix a positive integer $n$, and let $[n]:= \{1, 2, \cdots , n\}$. To any subset $S \subseteq [n]$, we associate the \textit{characteristic vector}:
\[
\chi_{S} = (\chi_{S}(1), \chi_{S}(2), \ldots, \chi_{S}(n))
\]
where 
\[
\chi_{S}(i) = 
\begin{cases} 
1, & i \in S \\
0, & i \notin S
\end{cases}
\]

Given a simplicial complex $\Delta$ over $[n]$, we associate to each face $\tau \in \Delta$ its characteristic vector $\chi_{\tau}$ and get the characteristic polytope as follows:

\begin{defi}
    The \textbf{characteristic polytope} $P_{\chi}(\Delta) \subset \mathbb{R}^{n}$ of $\Delta$ is the convex hull of its characteristic vectors:
\begin{center}
    $P_{\chi}(\Delta) = \text{conv} \{\chi_{\tau} \mid \tau \in \Delta\}$.
\end{center}
\end{defi}

\subsection{Characteristic polytope of a graph}
Given a graph $G$ with vertices $[n]$, and we view the graph as a simplicial complex. For any graph we can build a characteristic polytope as defined above. In what follows we will first do a few examples and then focus on two graphs, namely the complete graph $K_n$ and the cycle graph $C_n$. 

\subsubsection{Example: Complete graph $K_n$}
The complete graph $K_n$ has all subsets of $[n]$ as its faces. The characteristic polytope of $K_n$ is given by:
\[P_{\chi}(K_n) = \text{conv} \{\chi_S \mid S \subseteq [n]\}.\]

\subsubsection{Example: Cycle graph $C_n$}
The cycle graph $C_n$ has faces corresponding to all subsets of vertices that induce a connected subgraph. The characteristic polytope of $C_n$ is given by:
\[P_{\chi}(C_n) = \text{conv} \{\chi_S \mid S \subseteq [n], \text{ induces a connected subgraph}\}.\]

\subsection{Volume and Ehrhart series}
We are mainly interested in the volume and Ehrhart series of the characteristic polytopes. We list the volume and Ehrhart series of $n$-gon up to 5-gon as follows:
\begin{itemize}
    \item 3-gon: 
    \begin{align*}
    \text{vol }P_{\chi}(Q_3) &= 1, \\
    \text{Ehr}_{P_{\chi}(Q_3)}(z) &= \frac{1 + 4z + z^2}{(1 - z)^3}.
    \end{align*}


\end{document}