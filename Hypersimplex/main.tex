\documentclass[12pt]{article}

\usepackage[T1]{fontenc}
\usepackage{mathpazo}
\usepackage{eulervm}

\usepackage{amsmath}
\usepackage{amsthm}
\allowdisplaybreaks[4]
\usepackage{amssymb}

\usepackage{tikz}
\usepackage{tikz-3dplot}
\usetikzlibrary{decorations.pathreplacing, matrix}
\usepackage{nicematrix}
\usetikzlibrary{decorations.pathreplacing}
\usepackage[all]{xy} % Load xy package with 'all' option only once and before any package that might load it internally

\usepackage{bm}
\usepackage{dcolumn} %  This package is for aligning numbers in columns, useful for tables.
\usepackage[mathscr]{eucal} % This package provides script fonts for mathematical calligraphic symbols.
\usepackage{float}  % For float placement control (e.g., [H] for "exactly here")
\usepackage{graphicx} % For including images

\usepackage{times} % This package is deprecated. Consider using a different font package if needed.
\usepackage{epstopdf} % For including EPS images (usually not needed with pdflatex)

\usepackage[square]{natbib} % Enables Author-Year citation style
\usepackage{bibentry}
\usepackage{cite} % For grouped citations like (1, 2, 3)


\usepackage{youngtab}  % For Young tableaux
\usepackage{ytableau}
\ytableausetup{mathmode, boxsize=0.9em}

\setlength{\evensidemargin}{0.3cm} % Consider unifying these lengths
\setlength{\oddsidemargin}{0.3cm}  % Consider unifying these lengths
\parskip=6pt
\frenchspacing
\textwidth=15cm
\textheight=23cm
\parindent=16pt
\topmargin=-1.2cm

% Theorem styles
\theoremstyle{definition}
\newtheorem{thm}{Theorem}[subsection]
\newtheorem{defi}[thm]{Definition}
\newtheorem{lemma}[thm]{Lemma}
\newtheorem{prop}[thm]{Proposition}
\newtheorem{coro}[thm]{Corollary}
\newtheorem{conj}[thm]{Conjecture}
\newtheorem*{pf}{Proof}

\newtheorem{ex}[thm]{Example}
\newtheorem{remark}[thm]{Remark}

\newtheorem{prob}{Problem}[subsection]

\numberwithin{equation}{subsection}

%\usepackage{hyperref}  % Load hyperref LAST (important!)


%-------------------------------------------------------------
\begin{document}


\begin{center}
{\Large\bf 
Notes on Unimodality of Hypersimplex
%and 
%Combinatorial Classes for \\[10pt]
%Restrictions of a Hyperplane Arrangement
}\\ [7pt]
\end{center}

\vskip 3mm

\begin{center}
Mingzhi Zhang
\end{center}

\vskip 3mm

\section{Introduction}

\subsection{Hypersimplex}
Fix a positive integer $n$, and let $[n]:= \{1, 2, \cdots , n\}$. To any subset $S \subseteq [n]$, we associate the indicator vector:
\[
\chi_{S} = (\chi_{S}(1), \chi_{S}(2), \ldots, \chi_{S}(n))
\]
where 
$\chi_{S}(i) = 
\begin{cases} 
1, & i \in S \\
0, & i \notin S
\end{cases}$.
For $0 < k < n$, let $\binom{[n]}{k}$ be the family of all $k$-subsets of $[n]$. The \textit{hypersimplex} $\Delta_{k,n} \subseteq \mathbb{R}^n$ is the convex hull of the indicator vectors $\chi_I$ for $I \in \binom{[n]}{k}$. Equivalently, it can be defined as
\[
\Delta_{k,n} = \{(x_1, x_2, \cdots, x_n) \mid 0 \leq x_1, x_2, \cdots, x_n \leq 1; x_1 + x_2 + \cdots + x_n = k\}
\]


% Triangulations by Stanley
\subsection{Stanley's Triangulations}
Why the set defined by a permutation
$$
\nabla_w = \{(y_1,\dots,y_{n-1})\in [0,1]^{n-1} \mid 0 < y_{w(1)} < y_{w(2)} < \dots < y_{w(n-1)} < 1\}
$$
is indeed a simplex?

Step-by-Step Explanation

\begin{enumerate}
    \item What is a simplex? A simplex is the simplest generalization of a triangle or tetrahedron to higher dimensions:
    \begin{itemize}
        \item 1-simplex: line segment (2 vertices)
        \item 2-simplex: triangle (3 vertices)
        \item 3-simplex: tetrahedron (4 vertices)
        \item In general, an $(n-1)$-simplex in $\mathbb{R}^{n-1}$ is defined as the convex hull of exactly $n$ vertices, where these vertices are affinely independent (no vertex is a linear combination of the others with coefficients summing to 1).
    \end{itemize}
Formally, a simplex in $\mathbb{R}^{d}$ is the set
$$
\text{conv}(v_0, v_1, \dots, v_d) = \{ \lambda_0 v_0 + \lambda_1 v_1 + \dots + \lambda_d v_d \mid \lambda_i \ge 0, \sum_{i=0}^d \lambda_i = 1 \}.
$$
    \item Connecting our definition to a simplex. Our set is defined by ordering coordinates strictly:
$$
0 < y_{w(1)} < y_{w(2)} < \dots < y_{w(n-1)} < 1
$$
This set describes an "increasing path" from 0 up to 1 through all coordinates ordered by permutation $w$.

Crucially, the closure of this set is given by allowing equality:
$$
0 \leq y_{w(1)} \leq y_{w(2)} \leq \dots \leq y_{w(n-1)} \leq 1
$$
The closure of the set $\nabla_w$ includes points where coordinates equal 0 or 1 and where multiple coordinates might equal each other.
    \item Identifying the vertices of $\nabla_w$. The vertices of our simplex (the points we "move between") come from setting some subset of coordinates explicitly to either 0 or 1, following the strict inequalities. Specifically, consider the points formed by "corners" of the cube, where all coordinates are either 0 or 1.

Start by examining the set of points:
  $$
  v_0=(0,0,\dots,0), \quad v_1=(1,1,\dots,1), \quad v_2,\dots,v_{n-1}
  $$

Other vertices $v_2,\dots,v_{n-1}$ are points obtained by moving from 0 to 1, coordinate by coordinate, respecting the given permutation $w$. Precisely, the vertices are given by turning coordinates from 0 to 1 one-by-one following the order defined by $w$.

Thus, the vertices explicitly are the points:
$$
(0,0,\dots,0),\quad e_{w(1)},\quad e_{w(1)}+e_{w(2)},\quad \dots,\quad e_{w(1)}+e_{w(2)}+\dots+e_{w(n-1)}
$$
where $e_i$ is the unit vector in coordinate $i$. Notice that the last vertex is indeed the vector $(1,1,\dots,1)$.

There are clearly exactly $n$ points, precisely what we need for a simplex in dimension $n-1$.
    \item Affine independence of these vertices. To be sure we have a simplex, we must confirm affine independence of the vertices: affine independence intuitively means no vertex lies in the "flat" formed by the other vertices.

Here, the vertices are carefully chosen so that each step strictly "moves upward" along a different coordinate axis: Start at $(0,\dots,0)$.
Move in the direction of coordinate $w(1)$ first, then coordinate $w(2)$, and so on, until all coordinates become 1.

Because each step introduces a new coordinate that was previously 0, each vertex introduces a new dimension (coordinate) independent of the previous vertices. Thus, each vertex is adding a completely independent "direction".

Therefore, these vertices are clearly affinely independent. No vertex can be represented as an affine combination of the others.
    \item Geometric intuition. Geometrically, the set described by the inequalities:
  $$
  0 < y_{w(1)} < y_{w(2)} < \dots < y_{w(n-1)} < 1
  $$
represents a "slice" of the cube defined by a strict ascending order of coordinates, forming a "staircase-like" path in higher dimensions.

This "slice," when considered with its boundary, precisely matches the convex hull of the vertices we described, thus forming a simplex.

This simplex shape is the "natural" generalized triangle (or tetrahedron in 3D) formed by stepping from $(0,\dots,0)$ to $(1,\dots,1)$ along axes in a strictly ordered manner.

\end{enumerate}

Summary (Why this is a simplex):
\begin{enumerate}
    \item The set defined by permutation ordering naturally identifies vertices at corners of the cube (coordinates either 0 or 1).
    \item There are exactly $n$ vertices, perfect for an $(n-1)$-simplex.
    \item The vertices are affinely independent because each one introduces a new coordinate direction.
\end{enumerate}
Thus, the set $\nabla_w$ is a simplex.


% Volume and Ehrhart
\subsection{Volume and Ehrhart series}
We are mainly interested in the volume and Ehrhart seires of the characteristic polytopes. 
We list the volume and Ehrhart series of $n$-gon up to 5-gon as follows:
\begin{itemize}
    \item 3-gon: 
    \begin{align*}
    \text{vol }P_{\chi}(Q_3) &= 1, \\
    \text{Ehr}_{P_{\chi}(Q_3)}(z) &= \frac{1 + 4z + z^2}{(1 - z)^3}.
    \end{align*}
    \item 4-gon: 
    \begin{align*}
    \text{vol }P_{\chi}(Q_4) &= \frac{1}{2}, \\
    \text{Ehr}_{P_{\chi}(Q_4)}(z) &= \frac{1 + 5z + 5z^2 + z^3}{(1 - z)^4}.
    \end{align*}
    \item 5-gon:
    \begin{align*}
    \text{vol }P_{\chi}(Q_5) &= \frac{5}{24}, \\
    \text{Ehr}_{P_{\chi}(Q_5)}(z) &= \frac{1 + 6z + 11z^2 + 6z^3 + z^4}{(1 - z)^5}.
    \end{align*}
\end{itemize}

From the examples we see unimodality in the $h^*$-polynomial, and it is natural to conjecture that:
\begin{conj}
    The coefficients $h^*_{i}$ in the $h^*$-polynomial of the Ehrhart series of characteristic polytopes of $n$-gon form a unimodal sequence.
\end{conj}

\section{Characteristic Polytope of $d$-cube $\square_{d}$}




\begin{thebibliography}{99}
\bibitem[ABD10]{ABD10} Ardila, Federico, Carolina Benedetti, and Jeffrey Doker. "Matroid polytopes and their volumes." Discrete \& Computational Geometry 43 (2010): 841-854.

\bibitem[AHK18]{AHK18} Adiprasito, Karim, June Huh, and Eric Katz. "Hodge theory for combinatorial geometries." Annals of Mathematics 188.2 (2018): 381-452.

\bibitem[Ehr62]{Ehr62} Ehrhart, Eugene. "Sur les polyèdres rationnels homothétiques à n dimensions." CR Acad. Sci. Paris 254 (1962): 616.

\bibitem[Pos09]{Pos09} Postnikov, Alexander. "Permutohedra, associahedra, and beyond." International Mathematics Research Notices 2009.6 (2009): 1026-1106.

\end{thebibliography}


\end{document}