%-------------------- Preamble --------------------
\documentclass[12pt]{article}

% Basic fonts and math packages
\usepackage[T1]{fontenc}
\usepackage{mathpazo}
\usepackage{eulervm}
\usepackage{amsmath, amsthm, amssymb}
\allowdisplaybreaks[4]

% Graphics和tables
\usepackage{tikz}
\usepackage{tikz-3dplot}
\usetikzlibrary{decorations.pathreplacing, matrix}
\usepackage{nicematrix}
\usepackage[all]{xy}
\usepackage{graphicx}
\usepackage{float}
\usepackage{dcolumn}

% Other useful packages
\usepackage{bm}
\usepackage[mathscr]{eucal}
\usepackage{youngtab}
\usepackage{ytableau}
\ytableausetup{mathmode, boxsize=0.9em}
\usepackage{hyperref}
\usepackage{cleveref}

% 页面设置
\setlength{\evensidemargin}{0.3cm}
\setlength{\oddsidemargin}{0.3cm}
\parskip=6pt
\frenchspacing
\textwidth=15cm
\textheight=23cm
\parindent=16pt
\topmargin=-1.2cm

%-------------------- 定理环境 --------------------
\theoremstyle{definition}
\newtheorem{thm}{Theorem}[subsection]
\newtheorem{defi}[thm]{Definition}
\newtheorem{lemma}[thm]{Lemma}
\newtheorem{prop}[thm]{Proposition}
\newtheorem{coro}[thm]{Corollary}
\newtheorem{conj}[thm]{Conjecture}
\newtheorem*{pf}{Proof}
\newtheorem{ex}[thm]{Example}
\newtheorem{remark}[thm]{Remark}
\newtheorem{prob}{Problem}[subsection]
\numberwithin{equation}{subsection}

%-------------------- 正文开始 --------------------
\begin{document}

%-------------------- 标题与作者 --------------------
\begin{center}
    {\Large\bf Notes on Unimodality of Hypersimplex}
\end{center}
\vskip 3mm
\begin{center}
    Mingzhi Zhang
\end{center}
\vskip 3mm

%-------------------- 1. 引言 --------------------
\section{Introduction}

%---- 1.1 Hypersimplex ----
\subsection{Hypersimplex}
Fix a positive integer $n$, and let $[n]:= \{1, 2, \cdots , n\}$. To any subset $S \subseteq [n]$, we associate the indicator vector:
\[
\chi_{S} = (\chi_{S}(1), \chi_{S}(2), \ldots, \chi_{S}(n))
\]
where 
\[
\chi_{S}(i) = 
\begin{cases} 
1, & i \in S \\
0, & i \notin S
\end{cases}
\]
For $0 < k < n$, let $\binom{[n]}{k}$ be the family of all $k$-subsets of $[n]$. The \textit{hypersimplex} $\Delta_{k,n} \subseteq \mathbb{R}^n$ is the convex hull of the indicator vectors $\chi_I$ for $I \in \binom{[n]}{k}$. Equivalently,
\[
\Delta_{k,n} = \left\{ (x_1, \ldots, x_n) \in \mathbb{R}^n \mid 0 \leq x_i \leq 1,\, \sum_{i=1}^n x_i = k \right\}
\]

%---- 1.2 Volume and Ehrhart series ----
\subsection{Volume and Ehrhart series}
The volume of the hypersimplex $\Delta_{k,n}$ is given by the Eulerian number $A_{n-1,k-1}$ divided by $(n-1)!$. The Ehrhart series of $\Delta_{k,n}$ is given by:
\[
\text{Ehr}(\Delta_{k,n}, t) = \frac{\sum_{j=0}^{k-1} A_{n,j} t^j}{(1-t)^n}
\]
where $A_{n,j}$ are the Eulerian numbers.

%---- 1.3 Stanley's Triangulations ----
\subsection{Stanley's Triangulations}
\subsubsection{The Original Triangulation}
Let's break down the original idea that Stanley used to bridge the triangulations of polytopes $R_{n,k}$ and $S_{n,k}$ whose definitions will be given below.

\textbf{Definition of $R_{n,k}$ and $S_{n,k}$.}
\begin{itemize}
    \item $R_{n,k}$ is the \textit{regular} hypersimplex, defined as:
    \[
    R_{n,k} = \left\{ (x_1, \ldots, x_n) \in [0,1]^n \mid 0 \leq x_i \leq 1,\, \sum_{i=1}^n x_i = k \right\}
    \]
    \item $S_{n,k}$ is the hypersimplex with $k$ rises (with the convention that $y_0 = 0$):
    \[
    S_{n,k} = \left\{ (y_1, \ldots, y_n) \in [0,1]^n \mid y_0 = 0,\ \left|\left\{ i : y_{i-1} < y_i \right\}\right| = k \right\}
    \]
\end{itemize}

Clearly, $R_{n,k}$ is the hypersimplex $\Delta_{k,n}$ we defined above, while $S_{n,k}$ is a polytope where we count the number of rises in the coordinates. Both have the same volume which is equal to $\frac{A_{n,k}}{n!}$, where $A_{n,k}$ is the Eulerian number counting the number of permutations of $[n]$ with $k$ rises.

Before we proceed to the triangulation, let's clarify with an example what the polytope $S_{n,k}$ looks like. For $n=3$ and $k=2$, interior of the polytope $S_{3,2}$ consists of points $(y_1, y_2, y_3)$ with 2 rises, i.e.:
\[
\begin{array}{l}
0 = y_0 < y_1 < y_3 < y_2 < 1,\quad 0 = y_0 < y_3 < y_1 < y_2 < 1, \\
0 = y_0 < y_2 < y_1 < y_3 < 1,\quad 0 = y_0 < y_2 < y_3 < y_1 < 1.
\end{array}
\]
What we get is actually a triangulation of the polytope $S_{3,2}$. It is natural to index each of these simplices by the permutation of $[3]$ that gives the order of the coordinates, i.e., the first one corresponds to the identity permutation $(1,2,3)$, the second one corresponds to the permutation $(3,1,2)$, and so on. We will come back to this idea later in Lam and Postnikov's reformulation of the triangulation.


\subsubsection{The Volume-Preserving Map}
The question Stanley tried to answer is whether there is a map from $S_{n,k}$ to $R_{n,k}$ that preserves the volume. The answer is yes, and the map is given by:
\[
\begin{aligned}
    \phi: \quad & S_{n,k} \longrightarrow R_{n,k} \\
    & (y_1, \ldots, y_n) \longmapsto (x_1, \ldots, x_n)
\end{aligned}
\]
where
\[
x_i =
\begin{cases}
y_{i-1} - y_i, & y_i < y_{i-1} \\
1 + y_{i-1} - y_i, & y_i > y_{i-1}.
\end{cases}
\]
This map is piecewise linear, and we can see from the map that the $x_i$ coordinates record the rises (ascents) or falls (descents) of the $y_i$ coordinates. By two basic facts in permutation theory: (1) the number of rises in a permutation (with the convention that the first element is always a rise) plus the number of its falls is always $n$, and (2) the number of rises in a permutation equals the number of falls of its reverse, we can infer that the coordinates in the polytope $S_{n,k}$ have $n - 1 - (n - k) = k-1$ descents.

\subsubsection{Lam and Postnikov's Reformulation}

Lam and Postnikov~\cite{LP07} reformulated the triangulation of $S_{n,k}$ in terms of permutations. Consider the polytopes 
\[
\nabla_w = \{(y_1,\dots,y_{n-1})\in [0,1]^{n-1} \mid 0 < y_{w(1)} < \cdots < y_{w(n-1)} < 1\}
\]
where $w$ is a permutation of $[n-1]$. The question is: Is $\nabla_w$ a simplex?

\paragraph{Step-by-Step Explanation}
\begin{enumerate}
    \item \textbf{What is a simplex?} 
    A simplex is the simplest generalization of a triangle or tetrahedron to higher dimensions:
    \begin{itemize}
        \item 1-simplex: line segment (2 vertices)
        \item 2-simplex: triangle (3 vertices)
        \item 3-simplex: tetrahedron (4 vertices)
        \item In general, an $(n-1)$-simplex in $\mathbb{R}^{n-1}$ is the convex hull of $n$ affinely independent points.
    \end{itemize}
    Formally,
    \[
    \text{conv}(v_0, \ldots, v_d) = \left\{ \sum_{i=0}^d \lambda_i v_i \mid \lambda_i \ge 0,\, \sum_{i=0}^d \lambda_i = 1 \right\}
    \]
    \item \textbf{Connecting our definition to a simplex.} \\
    The set is defined by strict inequalities:
    \[
    0 < y_{w(1)} < \cdots < y_{w(n-1)} < 1
    \]
    The closure allows equality:
    \[
    0 \leq y_{w(1)} \leq \cdots \leq y_{w(n-1)} \leq 1
    \]
    \item \textbf{Identifying the vertices.} \\
    The vertices are:
    \[
    (0,0,\dots,0),\quad e_{w(1)},\quad e_{w(1)}+e_{w(2)},\quad \dots,\quad (1,1,\dots,1)
    \]
    where $e_i$ is the $i$-th unit vector.
    \item \textbf{Affine independence.} \\
    Each vertex introduces a new coordinate direction, so they are affinely independent.
    \item \textbf{Geometric intuition.} \\
    The region is a "slice" of the cube, forming a staircase-like path, which is a simplex.
\end{enumerate}

\paragraph{Summary}
\begin{enumerate}
    \item The set identifies $n$ vertices at cube corners.
    \item There are $n$ vertices, forming an $(n-1)$-simplex.
    \item Vertices are affinely independent.
\end{enumerate}
Thus, $\nabla_w$ is a simplex.



%-------------------- References --------------------


\begin{thebibliography}{1}

\bibitem{LP07}
T. Lam and A. Postnikov, "Alcoved polytopes I," Discrete and Computational Geometry, vol. 38, no. 3, pp. 453--478, 2007.

\end{thebibliography}

\end{document}
